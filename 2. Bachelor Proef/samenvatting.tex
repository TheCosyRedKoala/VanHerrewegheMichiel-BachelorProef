%%=============================================================================
%% Samenvatting
%%=============================================================================

% TODO: De "abstract" of samenvatting is een kernachtige (~ 1 blz. voor een
% thesis) synthese van het document.
%
% Een goede abstract biedt een kernachtig antwoord op volgende vragen:
%
% 1. Waarover gaat de bachelorproef?
% 2. Waarom heb je er over geschreven?
% 3. Hoe heb je het onderzoek uitgevoerd?
% 4. Wat waren de resultaten? Wat blijkt uit je onderzoek?
% 5. Wat betekenen je resultaten? Wat is de relevantie voor het werkveld?
%
% Daarom bestaat een abstract uit volgende componenten:
%
% - inleiding + kaderen thema
% - probleemstelling
% - (centrale) onderzoeksvraag
% - onderzoeksdoelstelling
% - methodologie
% - resultaten (beperk tot de belangrijkste, relevant voor de onderzoeksvraag)
% - conclusies, aanbevelingen, beperkingen
%
% LET OP! Een samenvatting is GEEN voorwoord!

%%---------- Nederlandse samenvatting -----------------------------------------
%
% TODO: Als je je bachelorproef in het Engels schrijft, moet je eerst een
% Nederlandse samenvatting invoegen. Haal daarvoor onderstaande code uit
% commentaar.
% Wie zijn bachelorproef in het Nederlands schrijft, kan dit negeren, de inhoud
% wordt niet in het document ingevoegd.

\IfLanguageName{english}{%
\selectlanguage{dutch}
\chapter*{Samenvatting}
\lipsum[1-4]
\selectlanguage{english}
}{}

%%---------- Samenvatting -----------------------------------------------------
% De samenvatting in de hoofdtaal van het document

\chapter*{\IfLanguageName{dutch}{Samenvatting}{Abstract}}

.NET MAUI (Multi-platform App UI) is een nieuw framework voor de ontwikkeling van cross-platform toepassingen die ingezet kan worden voor zowel de desktop als mobiele platformen. Het is ontwikkeld door Microsoft en komt voort uit het eerdere Xamarin.Forms framework. Met NET MAUI kunnen ontwikkelaars een enkele codebase schrijven en delen op verschillende platforms, waaronder iOS, Android, Windows en macOS.

Een bedrijf dat nog steeds gebruik maakt van een oud kassassysteem geschreven in COBOL, stootte afgelopen zomer op het probleem dat er geen floppydisks meer geproduceerd worden. Voor Goossens NV, het bedrijf in kwestie, is dit een groot probleem aangezien de gegenereerde dagrapporten van alle winkels op de bovengenoemde floppydisks geplaatst worden. Daarnaast willen ze het systeem uitbreidbaar maken met oog op de toekomst. Dit alles zorgt ervoor dat een ontwikkeling van een nieuw systeem zich aandringt.

Deze proef gaat na of .NET MAUI al voldoende ontwikkeld is om een real-time kassasysteem geprogrammeerd in COBOL te vervangen door een systeem geschreven in C\# gebruikmakend van het .NET MAUI-framework. De centrale onderzoeksvraag van deze proef luidt dan ook als volgt: ``Is .NET MAUI al voldoende matuur om een bestaande POS-systeem geschreven in COBOL te vervangen?''.

De uitvoering van dit onderzoek heeft als doel om het volledige systeem van een copycenter te simuleren. Dit systeem zal een analoog signaal van de printers moeten kunnen opvangen, omzetten in een digitaal signaal en doorgeven aan de grafische interface van het kassasysteem. Dit alles moet real-time gebeuren, wat wil zeggen dat vanaf het moment dat een digitaal signaal uitgezonden wordt, het kassasysteem zo goed als onmiddelijk zichzelf update met de nieuwe tellerstand. Daarnaast moet het ook mogelijk zijn om push-notificaties te versturen bij het afsluiten van het syteem. Het succes van de proef kan dan ook gedefinieerd worden op basis van twee criteria. De proef kan aanschouwd worden als een partieel succes wanneer de gemiddelde reactiesnelheid van de .NET MAUI-applicatie sneller is dan de gemiddelde reactiesnelheid van het huidige systeem. De proef is een geheel succes indien het versturen van push-notificaties mogelijk is met .NET MAUI.

Om een antwoord op de centrale onderzoeksvraag te kunnen formuleren, wordt een tweedelig onderzoek uitgevoerd. Het onderzoek bestaat uit een literatuurstudie, die als doel heeft context te schetsen, en een proof-of-concept, die bedoeld is om metingen op uit te voeren.

Uit de proof-of-concept blijkt dat de gemiddelde reactiesnelheid van .NET MAUI ongeveer drie en een half keer sneller is dan de reactiesnelheid van het huidige systeem. Daarnaast werd ook opgemerkt dat .NET MAUI geen algemene oplossing biedt om push-notificaties te versturen zonder platformspecifieke code te gebruiken.

De conclusie die getrokken kan worden uit bovenvermelde resultaten is dat deze proef een partieel succes is. Enerzijds is de gemiddelde reactiesnelheid van de .NET MAUI-applicatie sneller dan die van het huidige systeem, anderzijds is het niet mogelijk om push-notificaties te versturen zonder platformspecifieke code te gebruiken.
