%%=============================================================================
%% Inleiding
%%=============================================================================

\chapter{\IfLanguageName{dutch}{Inleiding}{Introduction}}%
\label{ch:inleiding}

\section{\IfLanguageName{dutch}{Probleemstelling}{Problem Statement}}%
\label{sec:probleemstelling}

Goossens NV, een bedrijf dat in gereviseerde printers doet, heeft ook enkele copycentra waar momenteel een zogenaamd \emph{''Point-Of-Sale-systeem''} draait. Dit is geschreven in de COBOL-programmeertaal. Aan het einde van elke dag wordt de kassa afgesloten en wordt door het systeem steeds een dagrapport gegenereerd en opgeslagen op een floppydisk. Echter aan het begin van de zomer stootte het bedrijf op het probleem dat er geen floppydisks meer geproduceerd werden. Daarnaast is het tegenwoordig ook verplicht om elektronische betalingen mogelijk te maken. Deze zaken zorgden er dan ook voor dat er nood was aan het moderniseren van het POS-systeem.

Dit onderzoek gaat na of het .NET MAUI framework van Microsoft al voldoende matuur is om het bestaande POS-systeem geschreven in COBOL te vervangen door een .NET MAUI-applicatie.

Het .NET MAUI framework werd gekozen voor dit onderzoek omdat dit framework enkele voordelen met zich meebrengt:

\begin{enumerate}
    \item Integratie
    
    .NET MAUI maakt deel uit van het .NET-ecosysteem. Dit laat toe om in de toekomst meer tools en/of applicaties te ontwikkelen die simpel te integreren zijn met het kassasysteem.
    
    \item Gebruikersinterface definitie met HTML
    
    Het .NET MAUI framework laat toe om de gebruikersinterface te definiëren met twee verschillende programmeertalen: XAML en HTML. Voor ontwikkelaars die gewend zijn van webapplicaties te ontwikkelen is de stap naar .NET MAUI aanzienlijk kleiner aangezien zij geen nieuwe taal moeten leren.
    
    \item Cross-platform
    
    Indien in de toekomst het kassasysteem op een mobiel platform zoals iOS of Android moet draaien, dan is dit mogelijk met een .NET MAUI applicatie. Dit framework laat toe om met één codebase applicaties te ontwikkelen voor meerdere platformen.
\end{enumerate}

\section{\IfLanguageName{dutch}{Onderzoeksvraag}{Research question}}%
\label{sec:onderzoeksvraag}

De onderzoeksvraag voor dit onderzoek luidt als volgt: is .NET MAUI al voldoende matuur om een bestaande POS-systeem geschreven in COBOL te vervangen.

Om dit te kunnen onderzoeken zijn er enkele subonderzoeksvragen die moeten leiden tot een conclusie op de hoofdvraag:

\begin{enumerate}  
    \item Wat is de gemiddelde reactiesnelheid van de .NET MAUI applicatie en hoe verhoudt deze zich ten opzichte van de gemiddelde reactiesnelheid van het huidige systeem?
    
    \item Laat .NET MAUI het toe om push-notificatie te versturen?
\end{enumerate}

\section{\IfLanguageName{dutch}{Onderzoeksdoelstelling}{Research objective}}%
\label{sec:onderzoeksdoelstelling}

Tijdens dit onderzoek zal een proof-of-concept opgezet worden die het volledige systeem van een copycenter simuleert. Tijdens het opzetten zal getracht worden om een python script te maken die het analoge signaal van een printer opvangt en omvormt naar een digitaal signaal met een payload. Ook zal nagegaan worden of .NET MAUI het toelaat om push-notificatie te versturen.

Daarna zullen metingen op de proof-of-concept en op het huidige systeem uitgevoerd worden. Deze zullen de gemiddelde reactiesnelheid applicaties peilen en vergelijken met elkaar.

De proef is een partieel succes indien:
\begin{itemize}
    \item De gemiddelde reactiesnelheid van de .NET MAUI-applicatie sneller is dan de gemiddelde reactiesnelheid van het huidige POS-systeem.
\end{itemize}

De gehele proef is een succes indien:

\begin{itemize}
    \item Het versturen van push-notificaties mogelijk is met .NET MAUI.
\end{itemize}

\section{\IfLanguageName{dutch}{Opzet van deze bachelorproef}{Structure of this bachelor thesis}}%
\label{sec:opzet-bachelorproef}

% Het is gebruikelijk aan het einde van de inleiding een overzicht te
% geven van de opbouw van de rest van de tekst. Deze sectie bevat al een aanzet
% die je kan aanvullen/aanpassen in functie van je eigen tekst.

De rest van deze bachelorproef is als volgt opgebouwd:

In Hoofdstuk~\ref{ch:stand-van-zaken} wordt een overzicht gegeven van de stand van zaken binnen het onderzoeksdomein, op basis van een literatuurstudie.

In Hoofdstuk~\ref{ch:methodologie} wordt de methodologie toegelicht en worden de gebruikte onderzoekstechnieken besproken om een antwoord te kunnen formuleren op de onderzoeksvragen.

In Hoofdstuk~\ref{ch:methodologie} wordt de opzet van de proof-of-concept besproken en uitgelegd welke metingen op deze proof-of-concept uitgevoerd zijn.

In Hoofdstuk~\ref{ch:conclusie}, tenslotte, wordt de conclusie gegeven en een antwoord geformuleerd op de onderzoeksvragen. Daarbij wordt ook een aanzet gegeven voor toekomstig onderzoek binnen dit domein.